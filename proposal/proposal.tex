\documentclass[12pt]{article}


\usepackage{amsthm}
\usepackage{amssymb}
\usepackage{amsmath}
\usepackage{tikz}
\usepackage{epsfig}
\usepackage{enumerate}
\usepackage{graphicx}
\usepackage{tabularx}
%%\usepackage[margin=0.5in]{geometry}

\usepackage{url}
\urlstyle{same}

%FlowChart Stuff
\usetikzlibrary{shapes,arrows}
\tikzstyle{decision} = [diamond, draw, text width=4.5em, text badly centered, node distance=3cm, inner sep=0pt]
\tikzstyle{block} = [rectangle, draw, text width=5em, text centered, rounded corners, minimum height=4em]

%%theorems and stuff
\newtheorem{theorem}{Theorem}[section]
\newtheorem{prop}[theorem]{Proposition}
\newtheorem{lemma}[theorem]{Lemma}
\newtheorem{result}[theorem]{Result}
\newtheorem{definition}[theorem]{Definition}

%%definitions
\theoremstyle{definition}
\newtheorem{example}{Example}[section]

%tab command
\newcommand{\tab}{\hspace*{2em}}

%%degree command
\newcommand{\degree}{\ensuremath{^\circ}}

%%Begin%%

\begin{document}

\title{Proposal}
\author{Wesley Bowman}
\date{\today}
\maketitle

%\begin{document}
    From the requirements, I would be able to assist on Model Tuning, Model
    Validation, Data Accessibility, Data Analysis, and possibly GIS interface.

    I proposal that I mostly work on anything requiring UTide. I would also
    like to be programming coordinator, making sure that no one is rewriting
    code that exist. This would require people to inform me as to the code they
    are writing and the purpose of that code. This could also come in the form
    of a Github page, that I could maintain. This would also help us realize
    where we currently are in the project, allowing us to know what needs to be
    done and what is already done.

    Essentially, be the gap between students and Karsten/Thomas, which would
    allow the students to have constant and instant
    communication about any questions.

    Most of the theory for validation I think will come from tidal
    constituents. Python is my preferred method of producing all of these
    requirements. Fortran would be a second go to if need be. In python, a
    standard list of used libraries would be:
    Pandas, netCDF4, numpy, scipy, matplotlib. From those, most everything can
    be done for this project (except for GIS, which would need more thought if
    we are doing that).

    Timeline would need to be discussed once we have a more detailed role
    definition for everyone, and which specific project everyone is working on.
    If multiple people are working on one project, then what in that project
    each individual is doing.

    I would like to finish UTide in it's entirety, and spruce it up a bit.



%\end{abstract}


\end{document}
