\documentclass[12pt]{article}


\usepackage{amsthm}
\usepackage{amssymb}
\usepackage{amsmath}
\usepackage{tikz}
\usepackage{epsfig}
\usepackage{enumerate}
\usepackage{graphicx}
\usepackage{tabularx}
%%\usepackage[margin=0.5in]{geometry}

\usepackage{url}
\urlstyle{same}

%FlowChart Stuff
\usetikzlibrary{shapes,arrows}
\tikzstyle{decision} = [diamond, draw, text width=4.5em, text badly centered, node distance=3cm, inner sep=0pt]
\tikzstyle{block} = [rectangle, draw, text width=5em, text centered, rounded corners, minimum height=4em]

%%theorems and stuff
\newtheorem{theorem}{Theorem}[section]
\newtheorem{prop}[theorem]{Proposition}
\newtheorem{lemma}[theorem]{Lemma}
\newtheorem{result}[theorem]{Result}
\newtheorem{definition}[theorem]{Definition}

%%definitions
\theoremstyle{definition}
\newtheorem{example}{Example}[section]

%tab command
\newcommand{\tab}{\hspace*{2em}}

%%degree command
\newcommand{\degree}{\ensuremath{^\circ}}

%%Begin%%

\begin{document}

\title{Proposal}
\author{Wesley Bowman}
\date{\today}
\maketitle

    1. Methodology:
        All programming will be done in Python, MatLab, or Fortran with it
        primarily being done in Python.

        \begin{itemize}
            \item Model Tuning: 

                I expect this be be done by Andy Balzer. The
                Python code will be contributed by me as he needs it. It will
                use the Python UTide on any data, and output it in a format
                that Andy would like. I also contributed to setting up the
                FVCOM run file for spatially-varying bottom frictions.

                These codes use the Python libraries netCDF4, numpy, pandas,
                and UTide primarily.

            \item Model Validation:

                I have written UTide in Python, so that model validation is
                possible with the harmonic constituents.


            \item Data Accessibility:

                This will be done using netCDF4 and pandas. There is a
                possibility of a GUI to be implemented here, upon request.
                This will not help on clusters though.

            \item Data Analysis:

                Using mpi4py along with netCDF4 and pandas, harmonic analysis
                of the elevation and velocities can be done and saved to an
                netCDF4 file in parallel. It will be written in parallel since
                the grid sizes can be large, and running UTide is an expensive
                process that can take a noticeable time.

                Iterpolation can be done using MatPlotLib, and pandas can be
                used to filter noise
                from the in-situ measurements.

            \item Format Conversion:

                Pandas allows for data to be easily manipulated into many
                different output formats. Between numpy, scipy, netCDF4, and
                pandas, all file formats should be able to be read in, and
                manipulated to the user specific output.

            \item Plotting:

                All plotting can be done using pandas with Matplotlib or just
                standalone matplotlib.

            \item GIS Interface:

                This will be done with python, accessing all other python code
                with the GIS server sending info to python, and python sending
                the correct data back. This may require arcpy, or it may be
                doable with no special libraries besides the ones mentioned
                above already.



        \end{itemize}

    2. Role Definition: I will be lead programming and programming coordinator.
    I will take on the difficult programming task, as well as assist others in
    their programming duties. I will also organize the programs so that
    everyone will be able to know where code is and if it is up to date. \\

    3. Detailed Time-line:

        I need a little more specifics before I can contribute to a timeline.
        UTide needs to be finished first, which I plan on doing in the entirety
        of next week. After that, we should be able to move onto all the
        components that require UTide to be finish, and I can also send it out
        for beta testing then. After that, a lot of code is already done, and
        needs to be correctly organized. Code that Aidan left needs to be
        sorted through to find all the useful bits, and after talking with
        Karsten, we may get Aidan to come down and give us a brief tour of his
        code to make tha a bit faster. Data organization should take around 3
        days I would imagine. After that, it will vary depending on the
        project. GIS interfacing will take a good while. The rest, each
        individual program should be given at least a 3 days to finish, that
        way the programmer is not rushed to finish and sends out an untested
        piece of code.



%    From the requirements, I would be able to assist on Model Tuning, Model
%    Validation, Data Accessibility, Data Analysis, and possibly GIS interface.
%
%    I proposal that I mostly work on anything requiring UTide. I would also
%    like to be programming coordinator, making sure that no one is rewriting
%    code that exist. This would require people to inform me as to the code they
%    are writing and the purpose of that code. This could also come in the form
%    of a Github page, that I could maintain. This would also help us realize
%    where we currently are in the project, allowing us to know what needs to be
%    done and what is already done.
%
%    Essentially, be the gap between students and Karsten/Thomas, which would
%    allow the students to have constant and instant
%    communication about any questions.
%
%    Most of the theory for validation I think will come from tidal
%    constituents. Python is my preferred method of producing all of these
%    requirements. Fortran would be a second go to if need be. In python, a
%    standard list of used libraries would be:
%    Pandas, netCDF4, numpy, scipy, matplotlib. From those, most everything can
%    be done for this project (except for GIS, which would need more thought if
%    we are doing that).
%
%    Timeline would need to be discussed once we have a more detailed role
%    definition for everyone, and which specific project everyone is working on.
%    If multiple people are working on one project, then what in that project
%    each individual is doing.
%
%    I would like to finish UTide in it's entirety, and spruce it up a bit.




\end{document}
